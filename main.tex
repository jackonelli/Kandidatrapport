\documentclass[12pt, a4paper]{article}
\usepackage[utf8]{inputenc}
\usepackage[T1]{fontenc}
\usepackage{authblk}
\usepackage{gensymb}
\usepackage{graphicx}
\usepackage{mathtools}
\usepackage{caption}
\usepackage{subcaption}
\usepackage{geometry}
\usepackage{cite}
\usepackage{color}
\newcommand{\HRule}{\rule{\linewidth}{0.5mm}}
\usepackage[nottoc]{tocbibind}
\usepackage[swedish]{babel}

\newenvironment{utkast}
{\color{red}} %Är din text preliminär infoga den i 
%\begin{utkast}
%TEXT
%\end{utkast}
%så ser alla att det stycket är just ett utkast. /Jakob

%\title{Planeringsrapport}
%\author{}
%\date{\today}

\begin{document}

\begin{titlepage}
%\noindent
\begin{flushleft}
\begin{minipage}{0.8\textwidth}
\includegraphics[width=\textwidth]{Chalmers.png}
\end{minipage}
%\begin{minipage}{0.6\textwidth}
%\textbf{\Huge CHALMERS}
%\end{minipage}
\\[0.5cm]
\HRule \\[7.0cm]

\end{flushleft}
\noindent
\textbf{{\Huge Agentbaserad modellering av \\
aktie- och optionsmarknad}}
\\[0.3cm]
{\large Kandidatarbete inom teknisk fysik}
\\[1.0cm]
Linnea Andersson \\
Sebastian Bertolino \\
Jakob Lindqvist \\
Per Nilsson Lundberg \\
Hanna Rasko \\
Samuel Rubenson \\[1.0cm]
\noindent
\HRule

\noindent
Institutionen för teknisk fysik \\
CHALMERS UNIVERSITY OF TECHNOLOGY \\
Göteborg, Sverige 2015
\end{titlepage}

\tableofcontents
\thispagestyle{empty}

\newpage
\setcounter{page}{1}
\section{Bakgrund}

% \textit{"Finanskrisen 2007-2008 och det faktum att den inte förutsågs och begränsades visade tydligt svagheten hos ekonomiska standardmodeller. Dessa är baserade på rationella välinformerade aktörer som gör att marknaden aldrig avviker långt från jämvikt, medan verkligheten innehåller komplikationer såsom nätverk av sammankopplade strukturer, begränsad information, positiva återkopplingsmekanismer och flockbeteenden." Text från Kandidatkatalogen för Teknisk Fysik 2015}

Grupper av individer tenderar klumpa ihop sig och ta beslut i flock. Sådana beslut baseras alltid på någon form av strategi som svar på given information. En strategi värderas efter dess förmåga att förutsäga rätt beslut, alltså det beslut som genererar mest värde. 

Det här kandidatarbetet modellerar en grupp individer som utgör ett slutet system där utfallet av varje enskild aktörs beslut bestäms av gruppens kollektiva beslut, helt isolerat från yttre faktorer.

Ett sådant system ger naturligt upphov till en växelverkan mellan olika strategier och vars dynamik är mer komplex än vad man väntar sig av ett så simpelt system.

\subsection{El Farol Bar-problemet}

1994 formulerade W. Brian Arthur El Farol Bar-problemet för att illustrera förloppet. Det är också detta problemet som ligger till grund för modellen i detta kandidatarbete:

Alla invånare i en småstad är sugna på att gå till lokala El Farol Bar. Problemet är bara att baren är så liten att man inte har roligt om, säg fler än 60\%, går dit samma kväll. Varje dag måste invånarna utan att kommunicera med varandra ta ett beslut huruvida de ska gå dit eller inte. Invånarna baserar sina strategier på att de får veta vilket beslut som varit gynnsamt för ett bestämt antal kvällar tillbaka, en historik. Sedan verkställs besluten, det fördelaktiga beslutet uppdateras i historiken och en ny runda av spelet börjar.

Det intressanta med problemet är att det givet en godtycklig historik inte finns någon globalt optimal strategi. Om det skulle finnas en sådan skulle alla välja den, fler än 60\% skulle gå till baren och samtliga skulle bli missnöjda. Avsaknaden av ett jämviktsläge där alla invånare är nöjda ger upphov till en mycket intressant dynamik trots spelets mycket enkla regler.

\subsection{Agentbaserad modellering: Makro från mikro}
Att komplexa system kan beskrivas fullt av sina minsta beståndsdelar är en tilltalande idé. Genom att reducera komplicerade förlopp till interaktioner mellan beståndsdelar \textit{agenter}, som ges enkla egenskaper kan hela systemet modelleras medelst simulering. Detta är fundamentet i den agentbaserade modelleringen.

Fördelarna är många jämfört med den klassiska makroskopiska modellen; agenterna har individuellt mycket enkla egenskaper och systemet påverkas snarare av hur de växelverkar. Komplexiteten begränsas alltså inte av analytiska metoder utan av datorkraft. I modellens enkelhet ligger dessutom en möjlighet att identifiera vilka basala parametrar som faktiskt styr systemets dynamik och med den statistiska fysikens verktyg kan de till och med förklaras analytiskt.

Agentbaserad modellering har väckt en del uppmärksamhet de senaste tio åren och har tillämpats i vitt skilda användningsområden, just för att den gör mycket komplexa system hanterbara. Till exempel används den inom den statistiska fysiken för att modellera spin-glas-tillstånd men också inom finansiell matematik vilket är fokuspunkten för det här kandidatarbetet.

\subsection{Matematiska och ekonomiska modeller}

Matematiska modeller av finansmarknader är ett instrument med vilket man dels värderar olika derivat men också använder till att förutsäga kursrörelser. Klassiska modeller så som Black-Scholes är bundna till antagandet om att kursrörelser drivs av en brownsk (normalfördelad) rörelse. Detta trots att faktisk finansiell data uppvisar fluktuationer som är större än en normalfördelning tillåter, Challet et al\cite{AnomalousFluctuations}. 

Rådande ekonomiska teorier bygger på antagandet att alla agenter är välinformerade och tar fullt rationella beslut, då dessa verkar på en marknad utan yttre störningar återvänder marknaden alltid till en jämvikt\cite{NeoClassicalEconomics}.

Agentbaserad modellering begränsas inte av brownsk rörelse och normalfördelning och har en principiell skillnad gentemot övriga ekonomiska teorier i det att agenterna varken är helt rationella eller perfekt informerade. De tar sina beslut baserade på en uppsättning strategier utifrån en begränsad tillgång till historik. Växelverkan mellan dessa strategier skapar flockbeteenden och bubblor, fenomen som uppkommer på riktiga marknader.

% Saker vi borde ha med som vi inte hade i plan.-rapport.:
%Stylized facts vad det nu kan vara.

\section{Syfte}


Vi vill främst undersöka tillämpningsmöjligheter på finansiella marknader varför modellens uppbyggnad och resultat kommer kopplas till en handelsmarknads byggstenar och aktörer. Förhoppningen är att genom modellering från ett mikroperspektiv kunna visa på makroeffekter som dels stämmer överens med empirisk data, men som även kan jämföras med klassiska matematiska modeller som har använts för att beskriva kursutvecklingar.

Vi vill undersöka om uppkomsten och omfattningen av tidigare finanskriser och bubblor kan analyseras med hjälp av vår modell. Enligt tidigare resultat inom agentbaserad modellering av minority games \cite{MarketCrashes} så uppkommer periodicitet i marknader, perioder som har sin början och får sitt avslut i en ''börskrasch''.

Genom att studera olika modus operandi för agenterna och hur det påverkar utfallet hoppas vi att vi kan förstå oss på hur aktörers handlingsmöjligheter eller begränsningar påverkar en marknad.




%Vi vill kunna identifiera likheter och beteenden för dagens aktie och optionsmarknad med hjälp av de olika teorierna vi ska prova... %utkast




\section{Problembeskrivning}

% \textit{Vi begränsar oss till aktiemarknaden där klassiska modeller förutsätter att aktiekurser i huvudsak följer en slumpvandring, vilket ger upphov till en normalfördelning av kursförändringarna. Det har visat sig att i den verkliga utvecklingen av aktiekurser är det betydligt vanligare med stora fluktuationer än förväntad från en normalfördelning. En metod som använts (inom området som kallats Econophysics) är agentbaserad modellering, där enkla beteenden hos samverkande individer på mikronivå kan ge upphov till komplexa fenomen på makronivå. Här kan t.ex. gruppbeteenden ge stora prisfluktuationer.}

Vi vill med detta kandidatarbete kunna analysera och dra slutsatser av beteende på aktiemarknaden för att möjligen kunna förutse olika händelser på marknaden. Klassisk finansiell matematik har modeller av aktiemarknaden (ex Black-Scholes) där kursen drivs av en Brownsk rörelse. Dessa modeller brister då det har observerats att det uppstår förändringar i marknadsvärdet med stora avvikelser mer frekvent än vad som förutspås av den normalfördelade modellen. Förhoppningen är att agentbaserad modellering med gruppbeteenden på mikronivån kan generera stora fluktuationer på makronivån. Drivkraften till ett sådant gruppbeteende är ett minoritetsspel.

\subsection{Minoritetsspel}
För att kunna utföra den här uppgiften behöver vi vara bekanta med det så kallade minoritetsspelet som går till på följande sätt: vid spelets början finns det $N$ st agenter, varje omgång väljer dessa agenter, oberoende av varandra, 1 eller 0. Gruppen som vinner är den som hamnar i minoritet. För att välja grupp har varje agent $s$ stycken olika strategier som talar om vilken grupp agenten ska välja beroende på de senaste $m$ utfallen. Mellan varje omgång delas poäng ut till de agenter som varit i minoritetsgruppen samt till de strategier som förutspått utfallet, oavsett om agenten valde den strategin eller ej. Agenterna väljer alltid den strategi med högst totalpoäng 

Denna enkla modell kan kopplas samman med en faktisk aktiemarknad om vi föreställer oss att de olika valen representerar köpa respektive sälja och vi tittar på storleken av minoritetsgruppen för varje tidssteg som ett sorts mått på vilka priser som skulle kunna uppstå. Det finns också en mer avancerad modell av minoritetsspelet som tillåter ett tredje val för agenterna som istället beror på hur framgångsrika dess strategier varit. Om denna tredje grupp räknas bort ifrån det totala antalet agenter så skapas en marknad där antalet agenter varierar mellan de olika tidsstegen och på så sätt simulerar en marknad där aktörer väljer att varken köpa eller sälja. 

 %Detta kommer vara vår grundtanke med denna studie. Vi kommer därefter vilja modifiera denna idé och komplettera med egenskaper för våra agenter. För att göra vår modell så lik verkligheten som möjligt och kunna utläsa händelser ur vår modell och möjligen jämföra den med verkligheten.

Här är vi intresserade av att skapa en sådan modell för att kunna extrahera data för att sedan jämföra med verkliga data och på så sätt se om det med hjälp av en sådan modell går att skapa simuleringar som kan förutsäga kommande utfall av marknaden. Men till detta syfte krävs det noga övervägande av vilka metoder som används för att styra agenternas handlande i modellen. %För att kunna göra detta krävs det att de delar som skiljer ett minoritetsspel från en verklig marknad övervägs noga.
\subsection{Intressanta aspekter}

För att skapa en modell av den kaliber som beskrivs i slutet av föregående sektion krävs mycket noga övervägande över på vilket sätt varje strategi poängsätts och värderas av agenterna. I verkligheten styrs aktörerna av hur mycket pengar de tjänar på sitt agerande. Om priset på en aktie är hög är det större chans att aktörerna vill sälja men mindre chans att de vill köpa och vice versa. Enligt detta kriterie är det viktigt att poängsättningen sker proportionellt mot den avtagande storleken på minoritetsgruppen. 

En annan viktig egenskap för systemet är hur verkställanden av vilka agenter som ska avstå hanteras. Det finns modeller där strategierna förutom att tilldelas poäng för framgångsrikt resultat också subtraheras poäng för när de är felaktiga och agenterna antas då avstå när ingen av deras strategier har tillräckligt höga poäng. Detta är då i kontrast till hur aktieägare/aktieköpare förväntas väga sina beslut på hur mycket en aktie beräknas vara värd vilket snarare skulle kunna modelleras som ett globalt pris som beräknas utifrån proportionerna mellan de olika grupperna.

Vidare så kräver en verklig marknad att agenten tidigare har köpt en aktie för att sedan kunna sälja den samt att det finns en villig köpare för varje säljare men i den här modellen antas alla agenter kunna handla som de vill. Vilket i verkligheten skulle innebära ett obegränsat antal aktier och därför skulle det vara av intresse att på något sätt införa en begränsning av handlandet vilket också skulle påverka priset som föreslogs i föregående stycke. 

Det skulle också vara intressant att försöka variera de olika agenternas egenskaper för att se om det går att dra fördel av att ha en annan längd på minnet eller ett annat antal strategier jämtemot de andra agenterna. Och på så sätt försöka finna en ultimat algoritm för hur en agents beslut ska tas i spelet. Detta vore dock inte direkt tillämpbart på en verklig marknad eftersom det skulle kräva vetskap av alla andra aktörers strategier. %kolla närmare på hur systemet påverkas av att agenter kan se olika långt bak i historien. Det finns flera sätt att ställa upp detta, både med att någon enstaka agent har längre/kortare minne än de andra eller om alla agenter har en slumpmässig längd på minnet för att se om någon och i så fall vilken part som skulle ha fördel av detta. 

%En stor skillnad är att agenterna här baserar sina val beroende på förutbestämda strategier medan verkliga aktörer kan tänkas basera sina val på ett aktivt medvetande som har möjlighet att överväga flera olika typer av information som till exempel vad den tror att andra aktörer kommer att göra i nästa steg vilket den här modellen bortser ifrån. 

%Till skillnad från en faktiskt marknad så saknas alternativet att avstå från aktion. Genom att lägga till något som representerar det alternativet skapas vad som kallas för ett storkanoniskt minoritetsspel där antalet agenter eller aktörer på marknaden kan variera för varje tidssteg. Denna modell kräver betydligt mer av systemet i och med att varje agent då måste ha någon form utav medvetande som talar om att nu är det inte lämpligt att agera vilket kan bero på flera olika saker. På en riktig marknad skulle det kunna ses som att agenten inte är intresserad av att agera då denne inte tjänar tillräckligt mycket på det. Detta kan modelleras med hjälp av en prissättning som är invers proportionell mot minoritetsgruppens storlek eller genom att agenter helt enkelt avstår om ingen av dess strategier har tillräckligt bra poäng. 

%  En annan möjlig parameter att ändra på är att de får valmöjlighet att inte delta om deras strategier historiskt sett har gett låg avkastning. Vid denna tillämpning, då antalet agenter varierar i spelet, kallas det för ett storkanonisk minoritetsspel. Då strategier får virtuella poäng även om dem inte används så kan en avstående agent alltid återvända till marknaden i framtiden om dess strategiers poäng förbättras. Det skulle även vara intressant att koppla vår modell till prissättning och att priset ändras när agenterna väljer att köpa/sälja. Agenterna skulle också kunna agera olika beroende på hur mycket pengar de har och vad priset för den underliggande tillgången är. Detta för att kunna koppla det till den verkliga marknaden.

% Det skulle vara intressant att ta hänsyn till att för en aktör ska kunna köpa måste en annan sälja. Även att ta i beaktning vad priset för köparen skulle vara. Ytterligare steg här skulle kunna vara att en agent då kan hålla en position under flera tidssteg. Det skulle vara väldigt omfattande för oss att tillämpa detta i vår modell. Därför ser vi detta som en avgränsning och istället något vi kan spåna vidare på vid tid över.

\subsection{Analys}

Simuleringar kommer påvisa makroskopiska effekter som i teorin kommer kunna jämföras med beteendet hos verkliga marknader. Genom att ändra modellen på mikronivå kan vi analysera hur dessa makroeffekter styrs av modellens parametrar.  



%Genom att extrahera och analysera data ifrån simuleringar av minoritetsspelet så är det möjligt att observera olika fysikaliska(?) fenomen. %Identifikation av sådana fenomen gör det möjligt att utläsa mer om systemets egenskaper och beteenden. Så kan analysen av dessa data hjälpa %förståelsen för inte bara systemets utan även den faktiska aktiemarknadens uppförande. 

 %istället för “är det möjligt att observera…” kanske “är målet att kunna få förståelse för verkliga marknadsfenomen"

%Den teoretiska delen av arbetet består av att 
Det är också av intresse att analytiskt undersöka modellen i ett försök att förstå vad som faktiskt driver processerna. Till exempel bör vi undersöka den periodicitet som uppkommer för ett minoritetsspel. Då agenterna spelar så kommer vissa strategier ge upphov till kedjor som återkommer. Dessa kedjor kan visuellt analyseras med hjälp av en De Bruijn-graf \cite{Periodicity}. Detta gör det även möjligt att studera metastrategier under spelets gång. Metastrategier är skilda från upprepade strategivinster genom att de förra försöker analysera tidigare agenters strategi (nuvarande paradigmet) för att få ett övertag. Det kräver då ett längre minne av agenten för att man ska kunna se periodcykler som visas av De Bruijn-grafen.

% ERSÄTT SOURCE http://en.wikipedia.org/wiki/De_Bruijn_graph




% startvärde, antal agenter, historia längd

% vad vill vi likna, ekonomi fysik




\section{Metod}

% \textit{Programmera (förslagsvis i Matlab) agentbaserad aktiehandel samt analysera det statistiska utfallet. Eventuellt köra simuleringarna på kluster (C3SE). Jämföra med empirisk data. Studera olika nivåer av komplexitet i agenternas modus operandi och hur det påverkar utfallet. Som förlängning eventuellt koppla till modellering av prissättning av optioner och hur denna avviker från standardmodellen (Black-Scholes) baserad på normalfördelade prisfluktuationer med konstant varians.}

%Bra metod! Vad tror du om en sådan här subsection-uppdelning för att få in något om skrivandet, Sebastian? Jag lägger in det på prov /Jakob
\subsection{Modellering}
%Jag tänker mig att vi kanske ska ha något om hur arbetet med modellen konkret ska delas upp./ Jakob 
Till att börja med kommer vi studera rapporter inom ämnet för att hitta information om hur minoritetsspelets grunder samt hur det fungerar. Men också vad forskare tidigare har gjort och kommit fram till. Dels för att se om vi kan återskapa deras tidigare resultat med hjälp av vår egna kod och dels för att få en riktlinje till hur vi kan implementera vår egen specifika modell av agenternas interaktion. För att smidigast kunna hantera de data som vi vill få fram kommer tillämpningen av modellen att ske i Matlab. Det ursprungliga minoritetsspelet kommer vara grunden i vår modell, sedan kommer vi successivt lägga till funktioner och egenskaper till den agentbaserade modelleringen för att först skapa en storkanonisk modell och sedan vår helt egen. Till exempel volym/likviditet, riktpris, metastrategier (dvs. ett medvetande om att historiken är periodisk) som kommer hjälpa oss analysera resultaten i form av vinnande agent, vinstperioder av strategier och periodtid.

En matematisk analys kommer också att göras för att få en teoretisk bakgrund till grundläggande koncept såsom förutsägbarhet och statistiska fördelningar men också mer fundamentala egenskaper ex. periodiciteten i historiken och avstånd i strategirummet. 

Följaktningsvis vill vi utveckla vår modell för att kunna lägga till olika beteenden till våra spelande agenter. Detta kommer ge oss många olika modeller och ett stort parameterrum som vi kanske kommer behöva köra på C3SE-klustret. Eftersom vi då kan köra flera %olika iterationer samtidigt och få mycket data kan vi senare med bra underliggande data ändra, korrigera och dra slutsatser om vår modell.
%Förslag på ändring: Nedanstående ersätter det bortkomment./Jakob
olika simuleringar för att dels kunna isolera enskilda parametrar och dels för att kunna få bra statistisk data.

%Jag lägger till någon bullshit här om skrivandeprocesser för att få ner Swenson i brygga. Det är verkligen bara ett utkast./Jakob
%Äh, det lutar nog att jag tar bort den här skiten nedan.
\subsection{Skrivande}
Rapporten är viktad som den enskilt viktigaste delen i arbetet och därför måste vi ha en klar bild av hur skrivandet ska gå till. Till att börja med startar vi skrivandet tidigt under arbetet, se tidsplanen. Redan nu i planeringsrapporten ställer vi upp klara syften och frågeställningar, genom att tidigt börja utreda dem bygger vi organiskt den struktur som den slutliga rapporten kommer ha. Revision av texten efter varje avslutat moment kommer ge ett iterativt skrivande och skapa en naturlig känsla för vad som faktiskt platsar i slutresultatet. 
Det kräver mycket av oss att börja skriva innan modelleringsarbetet är avslutat men det kommer vara en stor fördel att tidigt ha en fullständigt utkast att arbeta med. Vi kan individuellt ha större ansvar för olika delar av texten under förutsättning att det kontinuerligt sker en kollektiv granskning av materialet. Dessutom tänker vi anstränga oss för att maximalt utnyttja handledningen från fackspråk genom att under skrivandet samla på oss tankar och frågor.
%Ska vi ta med detta? 
%(Omformulera) Detta kommer också vare en av grundstenarna i våra redovisning då vi ska visa vad vi har gjort.




%Latex-Gant
\section{Tidsplan}


\begin{center}
\begin{table}[h!]

%\caption{\label{tidsplan}}
\begin{tabular}{| r| l| } \hline


 \emph{LP3 2015} & \emph{Vår plan} \\ \hline
  LV1 & Introduktion av projektet. \\& Färdigställa en enkel grundmodell.  \\ \hline
  LV2 & Börja med planeringsrapporten och utveckla grundmodellen \\& mot en storkanonisk modell \\ \hline
  LV3 & Avsluta utkast till planeringsrapport.\\ \hline
  LV4 & Bygga vidare på modellen och analysera resultat \\ \hline
  LV5 & Skapa ett skal för slutrapporten och anteckna erhållna resultat \\ \hline
  LV6 & Bygga vidare på modellen och analysera resultat \\ \hline
  LV7 & Bygga vidare på modellen och analysera resultat. \\& Handledningstillfälle 1. \\ \hline
  LV8 & Felsöka och bygga vidare på modellen. \\ \hline
  \emph{LP4 2015} &  \\ \hline
  LV1 & Fastställa avgränsningar och mål \\ \hline
  LV2 & Utveckla rapporten till ett första utkast \\ \hline
  LV3 & Ha klart grunden för vår analytiska analys \\ \hline
  LV4 & Slutliga jämförelser med empirisk data samt matematiska modeller \\ \hline
  LV5 & Färdigställa resultat \\ \hline
  LV6 & Sammanställning av rapporten \\ \hline  
  LV7 & Sammanställning av rapporten \\ \hline  
  LV8 & Opponering och redovisning av projektet \\ \hline
\end{tabular}
\caption*{Tidsplan}
\label{tidsplan}
\end{table}
\end{center} 
\newpage
\bibliographystyle{unsrt}
\bibliography{Referenser}
\end{document}
